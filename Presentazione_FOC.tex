\documentclass{beamer}
\usepackage[utf8]{inputenc}
\usepackage{listings} % Include the listings-package
\usepackage{graphicx}
\usepackage{booktabs}
\usepackage{xcolor}
\usepackage{wrapfig}
\usepackage{mathtools}
\usepackage{tikz-network}

\DeclarePairedDelimiter\abs{\lvert}{\rvert}%
\DeclarePairedDelimiter\bra{\langle}{\rvert}
\DeclarePairedDelimiter\ket{\lvert}{\rangle}
\DeclarePairedDelimiterX\braket[2]{\langle}{\rangle}{#1 \delimsize\vert #2}

\def\odiv{\, {\ominus\hspace{-8.5pt} \div} \,}
\definecolor{mygrey}{rgb}{0.1, 0.1, 0.1}
\usetheme{Montpellier}
\setbeamercolor{normal text}{fg=mygrey}

\title{Decomposition Structures for Soft Constraints Evaluation Problems}
\author{Laura Bussi}
\titlegraphic{\includegraphics[width=100pt,height=57pt]{marchio_unipi_pant541.png}}
\date{26/07/2019}
\institute[Università di Pisa]

\begin{document}

\frame{\titlepage}

\begin{frame}
\frametitle{Outline}
\begin{itemize}
	\item Introduction of an algebraic framework for representing and solving Soft 
	Constraint Evaluation Problems
	\begin{itemize}
		\item Syntax inspired by process algebras
	\end{itemize}
	\item Correspondence between tree decompositions and terms of the SCEP algebra
	\item Polyadic soft constraints as an interpretation of the given algebra
	\item Examples of application
\end{itemize}
\end{frame}

\section{Introduction}
 
\subsection{Constraint Satisfaction Problems and dynamic programming}
\begin{frame}
\frametitle{Constraint Satisfaction Problems and dynamic programming}
\begin{itemize}
	\item Constraint Satisfaction Problems can be represented as (hyper)graphs
	\begin{itemize}
		\item Intuitively, given a domain of values, a set of variables and a set of constraints,
		we want to find an assignment of the variables s.t. all the constraints are satisfied
		\item Further operations required to combine or remove constraints
		\item CSP solving is NP-complete
	\end{itemize}
\medskip
	\item Dynamic programming applies well to CSPs:
	\begin{itemize}
		\item We can decompose the problem and then repeatedly solve the subproblems
		\item Deciding the best decomposition is known as 
		\emph{secondary optimization problem}: NP-complete
	\end{itemize}
\end{itemize}
\end{frame}

\subsection{Soft constraints}
\begin{frame}
\frametitle{Soft constraints}
\begin{itemize}
	\item More flexible than classical constraints
	\begin{itemize}
		\item Each soft constraint associates a certain assignement to a value in a poset
		\item To solve an SCSP we seek for an assignement of the variable s.t. the total amount
		is minimized or maximized, due to the fact that we express negative or positive
		preferences
	\end{itemize}
\medskip
	\item SCSPs are difficult to represent and compose
	\begin{itemize}
		\item Need for a generalisation which allows for representing both the problem's
		structure and the solution process via an algebraic framework
		\item Decomposition and solving are thus correct by construction
	\end{itemize}
\end{itemize}
\end{frame}

\begin{frame}
\frametitle{Algebraic structures for soft constraints}
\begin{itemize}
	\item Absortive semirings (also known as c-semirings) 
	$\langle S, \oplus, \otimes, 0, 1 \rangle$
	\begin{itemize}
		\item The $\oplus$ operator is idempotent and 1 is its absorbing element
		\item The $\otimes$ operator is associative and commutative
	\end{itemize}
\medskip
	\item $\oplus$ is often used to induce the order on the elements of such a semiring
	\begin{itemize}
		\item We have $a \leq b \iff a \oplus b = b$
	\end{itemize}
\medskip
	\item We use $\otimes$ to combine constraints
\medskip
	\item We may additionaly require some kind of "subtraction"
	\begin{itemize}
		\item Based on residuation theory
		\item The residuation operator is a kind of weak inverse of $\otimes$
	\end{itemize}
\end{itemize}
\end{frame}

\begin{frame}
\frametitle{Algebraic structures for soft constraints}
\begin{itemize}
	\item Properties of semirings are useful for manipulating soft constraints
	\begin{itemize}
		\item Associativity and commutativity of $\otimes$ guarantee the order of combination
		to be irrelevant
		\item Absortivity enforces the fact that adding constraints decreases the number of
		solutions
		\item The existence of the $0$ element represents the crisp feature of total dislike
		of any solution involving a certain assignement
		\item The unit $1$ allows for modelling the crisp feature of "indifference" for a
		certain constraint
	\end{itemize}
\end{itemize}
\end{frame}

\begin{frame}
\frametitle{Polyadic Soft Constraints}
\begin{itemize}
	\item A formalism to manipulate soft constraints based on polyadic algebras
	\begin{itemize}
		\item Two families of operators, called \emph{cylindrification} and 
		\emph{polyadic substitution} model variables hiding and parameters passing
		\item Allows for a polynomial representation of soft constraints
	\end{itemize}
\medskip
	\item This algebra respects both the weak and the strong
	SCEP specification, thus representing an interpretation of the initial algebra
\end{itemize}
\end{frame}

\section{Networks and tree decomposition}

\subsection{Concrete and abstract networks}
\begin{frame}
\frametitle{Soft Constraints and concrete networks}
\begin{itemize}
	\item We assume $\mathcal{E}_C$ to be a ranked alphabet of soft constraints, equipped with
	an arity function $ar: \mathcal{E}_C \rightarrow \mathbb{N}$ and a set of variables 
	$\mathbb{V}$
\medskip
	\item A \emph{concrete network} is a pair $I \blacktriangleright N$ where
	\begin{itemize}
		\item $N = (V_N, E_N, a_N, lab_N)$ is a labelled hypergraph over $\mathcal{E}_C$
		with no isolated vertices
		\item Vertices are a subset of $\mathbb{V}$
		\item $I$ is a set of interface variables
	\end{itemize}
\medskip
	\item We also assume that $\mathcal{E}_C$ has a function 
	$var: \mathcal{E}_C \rightarrow \mathbb{V}^*$ which assigns a tuple of distinct canonical
	variables to each constraint, i.e. $var(A) \cap var(B) = \emptyset$ if $A \neq B$
\end{itemize}
\end{frame}

\begin{frame}
\frametitle{Example}
\begin{columns}
\begin{column}{0.5\textwidth}
   \begin{itemize}
	\item $A, B$ constraints with $ar(A) = ar(B) = 2$
	\item $var(A) = (x_1, x_2)$, $var(B) = (x_3, x_4)$
	\item $V_N = \{x,y,z\}$, $E_N = \{e_1,e_2\}$
	\item $a_N(e_1) = \langle x, y \rangle$, $a_N(e_2) = \langle y, z \rangle$
	\item $lab_N(e_1) = A$, $lab_N(e_2) = B$	
\end{itemize}
\end{column}
\begin{column}{0.5\textwidth}
	\begin{itemize}
		\item The concrete network $\{y\} \blacktriangleright N$:
	\end{itemize}
    \begin{center}
     \includegraphics[width=0.6\textwidth]{con_net.png}      
    \end{center}
\end{column}
\end{columns}
\end{frame}

\begin{frame}
\frametitle{Soft Constraint Satisfaction Problems}
\begin{itemize}
	\item A Soft Constraint Satisfaction Problem (SCSP) is a tuple 
	$(I \blacktriangleright N, \mathbb{D}, S, val)$
\medskip
	\item A \emph{value} $val_A$ is a function giving a cost in $S$ to each assignement of 
	canonical variables in $A$
	\begin{itemize}
		\item We will use $val_e$ as a function giving a cost to every assignement to variables
		$e$ is attached to
	\end{itemize}
\medskip
	\item Variables $I$ are those of interest
\medskip
	\item The solution is a function $sol: (I \rightarrow \mathbb{D}) \rightarrow S$: for each
	assignement $\rho: I \rightarrow \mathbb{D}$
	    \[ sol(\rho) = \bigoplus_{\{ \rho': V_N \rightarrow \mathbb{D} \mid 
		\rho' \downarrow_I = \rho \}} (\bigotimes_{\{i \mid e_i \in E_N \}}val_e(\rho' 
		\downarrow_{a_N(e_1)})) \]
\end{itemize}
\end{frame}

\begin{frame}
\frametitle{A canonical representative for networks}
\begin{itemize}
	\item Since we are interested only in interface variables, we can define isomorphisms
	between networks
		\item Then two networks are isomorphic if there is an isomorphism between them
		preserving the interface variables
		\item An abstract network $I \triangleright N$ is an isomorphism class of concrete
		networks
		\item In the following we assume the choice of a canonical representative of abstract
		networks
\end{itemize}
\end{frame}

\subsection{Tree decomposition}
\begin{frame}
\frametitle{Tree decomposition}
\begin{itemize}
	\item A decomposition of a graph can be represented as a tree decomposition
	\begin{itemize}
		\item Each vertex is a piece of the graph
	\end{itemize}
\medskip
	\item A (rooted) tree decomposition of a hypergraph G is a pair $\mathcal{T} = (T,X)$ where
	$T$ is a rooted tree and $X = \{ X_t\}_{t \in V_T}$ is a family of subset of $V_G$ s.t.
	\begin{itemize}
		\item For each vertex $v \in V_G$ there exists a vertex of $\mathcal{T}$ s.t.
		$v \in X_t$
		\item For each hyperedge $e \in E_G$ there is a vertex of $\mathcal{T}$ s.t. 
		$a_G(e) \subseteq X_t$
		\item Each subtree generated from $v \in V_G$ is a rooted tree
	\end{itemize}
\end{itemize}
\end{frame}

\subsection{Tree decomposition}
\begin{frame}
\frametitle{Tree decomposition}
\begin{itemize}
	\item The decomposition of a network $I \triangleright N$ is a decomposition of $N$ where we 
	choose as a root a vertex of $\mathcal{T}$ containing all the interface variables
\medskip
	\item We'll provide a translation from tree decompositions to SCEP (weak) terms: this will
	enable applying algebraic techniques to tree decompositions
\medskip
	\item We'll refer to \emph{completed versions} of tree decompositions, which explicitly
	associate components of $N$ to vertices of $\mathcal{T}$
	\begin{itemize}
		\item for each $v \in V_N (e \in E_N)$, $t_v (t_e)$ is the vertex closest to the root
		of $\mathcal{T}$ s.t. $v \in X_{t_v}$
	\end{itemize}
\end{itemize}
\end{frame}


\subsection{SCEP specification}
\begin{frame}
\frametitle{SCEP signature}
\begin{itemize}
	\item SCEP algebras are \emph{permutation algebras}
\medskip
	\item The SCEP-signature (\emph{s-signature}) is given by the following grammar:
	\begin{center}
		$p,q := p \mid \mid q \mid (x)p \mid p \pi \mid A(\widetilde{x}) \mid nil$
	\end{center}
\medskip
	\item The free variables of $p$ ($fv(p)$) are defined recursevly as expected
	\begin{itemize}
		\item the restriction $(x)$ is a bounding operator
		\item the free variables of $p \pi$ are just $\pi (fv(p))$
	\end{itemize}
\end{itemize}
\end{frame}

\begin{frame}
\frametitle{Strong SCEP specification}
\begin{itemize}
	\item A certain set of axioms toghether with the signature give the strong SCEP specification
	(\emph{s-specification})
	\begin{itemize}
		\item $\mid \mid$ forms a commutative monoid
		\item Restrictions can be swapped and $\alpha$-converted
		\item Permutations distribute over syntactic operators
	\end{itemize}		
\end{itemize}
\begin{center}
	\includegraphics[width=0.75\textwidth]{s-axioms.png}      
\end{center}
\end{frame}

\begin{frame}
\frametitle{Weak SCEP specification}
\begin{itemize}
	\item The axiom $AX_{SE}$ in the s-specification states that the scope of restricted
	variables can be narrowed to terms where they occur free
\medskip
	\item This cause s-terms with different decompositions to be equivalent
\medskip
	\item To distinguish different decompositions, we define a \emph{weak} SCEP specification
	(\emph{w-specification}) where $AX_{SE}$ is replaced with
	\begin{center}
		($AX_{(x)}^w$) \ $(x)p \equiv_w p$ \ \ $(x \not \in fv(p))$
	\end{center}
\end{itemize}
\end{frame}


\begin{frame}
\frametitle{Normal and canonical form}
\begin{itemize}
	\item W-terms can be seen as networks having a hierarchical structure
\medskip
	\item Normal and canonical forms are those of interest
\medskip
	\item A w-term is said to be in normal form whenever it is of the form
	$(\widetilde{x})(A_1(\widetilde{x_1}) \mid \mid ... \mid \mid A_n(\widetilde{x_n}))$
\medskip
	\item A w-term is said to be in canonical form whenever is obtained by the repeated
	application of ($AX_{SE}$) until termination
\medskip
	\item As we'll see, a term's form have an impact on its evaluation complexity
\end{itemize}
\end{frame}

\begin{frame}
\frametitle{Soundness and completeness of networks}
\begin{itemize}
	\item The s-specification is sound and complete w.r.t. networks
\medskip
	\item We can define a translation function which, given a concrete network 
	$I \blacktriangleright N$, gives us the corresponding s-term and viceversa
	\begin{itemize}
		\item $term(I \blacktriangleright N) = (V_N \setminus I) 
		(A_1(\widetilde{x_1}) \mid \mid ... \mid \mid A_n(\widetilde{x_n}))$
		\item $net(p) = fv(p) \blacktriangleright N_p$
	\end{itemize}
\medskip
	\item Completeness follows by the fact that if $net(n_1) \cong net(n_2)$, then
	$n_1 \equiv_s n_2$
\end{itemize}
\end{frame}

\begin{frame}
\frametitle{SCSPs as SCEPs}
\begin{itemize}
	\item SCSPs can be represented and solved as SCEPs
\medskip
	\item SCEPs are indeed more general than SCSPs
	\begin{itemize}
		\item There exist optimisation problems which are SCEPs and cannot be represented
		as SCSPs
	\end{itemize}
\medskip
	\item Thus we can define an algebra $\mathcal{V}$ s.t. 
	$[\![ I \triangleright N ]\!]^{\mathcal{V}}$ gives the solution of the SCSP defined
	over the network
\medskip
	\begin{itemize}
		\item[Theorem] Given an SCSP with underlying network $I \triangleright N$ and
		value functions $val_A$, we have that $I \triangleright N$ evaluated in $\mathcal{V}$
		($[\![ I \triangleright N ]\!]^{\mathcal{V}}$) is its solution
	\end{itemize}
\end{itemize}
\end{frame}

\subsection{Evaluation Complexity}

\begin{frame}
\frametitle{Evaluation Complexity}
\begin{center}
	\includegraphics[width=1\textwidth]{complexity.png}      
\end{center}
\begin{itemize}
	\item We now introduce a notion of complexity of w-terms ($\langle \langle - \rangle \rangle$)
\medskip
	\item The complexity of $p$ ($\langle \langle p \rangle \rangle$) is the maximum "size" of
	elements of an algebra $\mathcal{A}$ computed while inductively constructing 
	$[\![ p ]\!]^{\mathcal{A}}$
\medskip
	\begin{itemize}
		\item[Theorem] Given a term $p$, let $n$ be its normal form. For all canonical
		forms $c$ of $p$ we have $\langle \langle c \rangle \rangle \leq \langle \langle n
		\rangle \rangle$
	\end{itemize}
\end{itemize}
\medskip

\end{frame}

\subsection{Tree decompositions as w-terms}

\begin{frame}
\frametitle{Tree decompositions as w-terms}
	\begin{itemize}
		\item Given a network $I \triangleright N$, 
		let $\mathcal{CT} = (\mathcal{T}, \{ t_x \}_{x \in E_N \cup V_N})$ the completed
		version of one of its tree decompositions
	\medskip
		\item We translate $\mathcal{CT}$ into a w-term:
		\begin{itemize}
			\item Given $t \in T$, let $V(t) = \{ v \in V_N \mid t_v = t \}$ and 
			$E(t) = \{ e \in E_N \mid t_e = t \}$
			\item Suppose $t$ has children $t_1,...,t_n$ and $E(t) ) \{ e_1,...,e_k \}$
			\item Let $\widetilde{x} = V(t) \setminus I$
		\end{itemize}
	\medskip
		\item The term $\chi(t)$ is inductively defined as:
		\begin{center}
			$\chi(t) = (\widetilde{x})(A_1(\widetilde{x_1}) \mid \mid 
			... \mid \mid A_k(\widetilde{x_k}) \mid \mid \chi(t_1) \mid \mid 
			... \mid \mid \chi(t_n))$
		\end{center}
	\end{itemize}
\end{frame}

\begin{frame}
\frametitle{Tree decompositions as w-terms}
\begin{itemize}
	\item Given a tree decomposition of $\mathcal{T}$ rooted in $r$, the corresponding
	w-term $wterm(\mathcal{T})$ is $\chi(r)$ computed on the completed version 
	of $\mathcal{T}$
\medskip
	\item $wterm(\mathcal{T})$ correctly represents the network $\mathcal{T}$ decomposes:
	if $\mathcal{T}$ is a tree decomposition for $I \triangleright N$, then 
	$[\![ wterm(\mathcal{T}) ]\!]^{\mathcal{N}} = I \triangleright N$
\medskip
	\item Furthermore, given a tree decomposition $\mathcal{T}$, we have 
	$\langle \langle wterm(\mathcal{T}) \rangle \rangle \leq width(\mathcal{T})$
	
\end{itemize}
\end{frame}

\subsection{Computing Canonical Decomposition}

\begin{frame}
\frametitle{An algorithm for computing canonical decomposition}
\begin{itemize}
	\item Based on \emph{bucket elimination}
	\begin{itemize}
		\item Differently from it, the introduced algorithm produces all and only
		canonical terms
	\end{itemize}
\medskip
	\item Here putting a constraint in the bucket of its last variable means to apply
	the $(AX_{SE})$ axiom
\medskip
	\item The input are an s-term $(R)A$ in normal form and a total order $O_R$ over $R$
\medskip
	\item The output is a w-term $P$ in canonical form
\medskip
	\begin{itemize}
		\item[Theorem] $C$ is a canonical form of $R(A)$ if and only if there 
		is $O_R^C$ s.t. the algorithm with inputs $(R)A$ and $O_R^C$ outputs $C$
	\end{itemize}
\end{itemize}
\end{frame}

\begin{frame}
\frametitle{An algorithm for computing canonical decomposition}
\begin{center}
	\includegraphics[width=0.9\textwidth]{algorithm.png}
\end{center}
\end{frame}

\section{The algebra of Polyadic Soft Constraints}

\subsection{PSC as an interpretation of the initial algebra}

\begin{frame}
\frametitle{PSC as an interpretation of the initial algebra}
\begin{itemize}
	\item We define the algebra $\mathcal{P}$ of polyadic soft constraints
\medskip
	\item Constants are $A^{\mathcal{P}}(x_1,...,x_n)\eta = 
	c_{(x_1,...,x_n)}(\eta \circ \widehat{\sigma})$, \ $nil^{\mathcal{P}} = 0$
\medskip
	\item Operations are
	\begin{itemize}
		\item $((X)^{\mathcal{P}} c) \eta = (\exists_X c)\eta = 
		\bigvee_\rho \{ c\rho \mid \eta_{\mid_{\mathbb{V} 
		\setminus X}} = \rho_{\mid_{\mathbb{V} \setminus X}}\}$
		\item $c \pi^{\mathcal{P}} = (s_\pi c)\eta = c(\eta \circ \pi)$
		\item $c_1 \mid \mid^{\mathcal{P}} c_2 = (c_1\eta) \otimes (c_2\eta)$
	\end{itemize}
\medskip 
	\item $\widehat{\sigma}$ is used to map $var(A)$ to $\langle x_1, ..., x_n \rangle$
\end{itemize}
\end{frame}

\subsection{Weak and strong axioms for $\mathcal{P}$}

\begin{frame}
\frametitle{Weak and strong axioms for $\mathcal{P}$}
\begin{itemize}
	\item Axioms for $\alpha$-conversion and swapping of restrictions hold in $\mathcal{P}$,
	as well as those involving permutations and parallel composition
\medskip
	\item $\mathcal{P}$ is a w-algebra: it holds $(\exists_X c)\eta = c\eta$ 
	if $X \cap supp(c) = \emptyset$ ($AX_{(x)}^w$)
\medskip
	\item $\mathcal{P}$ is an s-algebra: it holds $X \cap supp(c2) = \emptyset \implies
	(\exists_X(c_1 \otimes c_2))\eta = (\exists_X(c_1 \otimes \exists_X c_2))\eta =
	(\exists_X c_1 \otimes \exists c_2)\eta = (\exists_X c_1)\eta \otimes (\exists_X c_2)\eta = 
	(\exists_X c_1)\eta \otimes c_2\eta$
\end{itemize}
\end{frame}

\section{Example}

%\begin{frame}
%\frametitle{Example}
%\begin{itemize}
%	\item Suppose we have two services, $A$ and $B$, accepting $n$ requests 
%\medskip
%	\item We have $m$ clients, each one associated with a certain probability to be served 
%	\begin{itemize}
%		\item We have independent probabilities
%		\item For each client, the probability to be served by $A$ or $B$ are different
%	\end{itemize}
%\medskip
%	\item Suppose that $k$ of the clients which ask for $A$'s services
%	ask for $B$'s services too, with $k < n$
%\medskip
%	\item We want to maximize the probability for a certain group to be served
%\end{itemize}
%\end{frame}&^

%\begin{frame}
%\frametitle{Example}
%\begin{itemize}
%	\item Two constraints: ${serve_A}_{(x_1,...,x_n)}$ and 
%	${serve_B}_{(x_1,...,x_k,x_{n+1},...,x_{n+k})}$
%\medskip
%	\item The domain is $\mathbb{N}$ and $S$ is the \emph{probabilistic} semiring 
%	$\langle [0,1], \leq, \times, 1 \rangle$
%\medskip
%	\item The clients for which we want to maximize the probabilities are $x_1,...,x_l$, 
%	and $x_{n+1},...,x_{n+l}$, $l < k$
%\medskip
%	\item An example network for $n = 3$:
%\end{itemize}
%\begin{center}
%	\includegraphics[width=0.55\textwidth]{concr_example.png}
%\end{center}
%\end{frame}

%\begin{frame}
%\frametitle{Example}
%\begin{itemize}
%	\item The problem in normal form is:
%\end{itemize}
%	\[ (\exists_{\{x_1',...,x_l'\}}\exists_{\{x_{n+1}',...,x_{n+l}'\}}
%	({serve_A}_{(x_1',...,x_n')} \times 
%	\exists_{\{x_1',...,x_l'\}}{serve_B}_{(x_1',...x_{n+k}')}))\eta\]
%\begin{itemize}
%	\item The problem in canonical form is:
%\end{itemize}
%	\[ (\exists_{\{x_1',...,x_l'\}}{serve_A}_{(x_1',...,x_n')})\eta \times 
%	(\exists_{\{x_1',...,x_l'\}}\exists_{\{x_{n+1}',...,x_{n+l}'\}}
%	{serve_B}_{(x_1',...x_{n+k}')})\eta\]
%\begin{itemize}
%	\item While in normal form we have $\langle \langle n \rangle \rangle = n+k$, the canonical
%	form allows for solving the two subproblems and then combine their solutions:
%	$\langle \langle c \rangle \rangle = n$
%\end{itemize}
%\end{frame}

\begin{frame}
\frametitle{Example}
\begin{itemize}
	\item Consider a network where meeting activities for a group require the existence of paths
	between every pair of collaborators
\medskip
	\item We assume that the network is composed of end-to-end two-way connections with
	independent probabilities of failure
\medskip
	\item We want to find the probability of a certain group to stay connected
\medskip
	\item We consider a graph with probabilities associated to each edge: the solution is the
	probability of some interface vertices staying connected
\end{itemize}
\end{frame}

\begin{frame}
\frametitle{Example}
\begin{itemize}
	\item We evaluate networks $I \triangleright N$ into an algebra of probability distributions
	on the partitions of $I$ ($Part(I)$)
\medskip
	\item If we are interested in a group $J$, we compute the probability distribution $P$ for
	$J \triangleright N$ and then we select $P(\{J\})$
\medskip
	\item Let us focus on parallel composition, defined as:
\end{itemize}
		\[ [\![ I_1 \triangleright N_1 \mid \mid I_2 \triangleright N_2 ]\!]^{\mathcal{D}}\Pi =
		\sum_{\Pi' \in Part(I \cup {x}) \mid \Pi'-x = \Pi} 
		[\![ I_1 \triangleright N_1 ]\!]^{\mathcal{D}}\Pi_1 \times
		[\![ I_2 \triangleright N_2 ]\!]^{\mathcal{D}}\Pi_2 \]
\begin{itemize}
	\item Here $\Pi \in Part(I_1 \cup I_2)$ and each $\Pi_1$, $P_2$ must belong to $Part(I_1)$
	and $Part(I_2)$ respectively. The $\cup$ operator produces the finest partition coarser than 
	the two components and $\times$ is the multiplication on reals
\end{itemize}
\end{frame}

\begin{frame}
\frametitle{Formal specification}
\begin{itemize}
	\item Considered networks are wheels $W_k(v,x)$
\medskip
	\item For $k = 2$ we have the following graph:
\end{itemize}
\begin{center}
	\includegraphics[width=1\textwidth]{wheel.png}      
\end{center}
\end{frame}

\begin{frame}
\frametitle{Non existence of an SCSP formulation}
\begin{itemize}
	\item The given SCEP does not fit the SCSP format
\medskip
	\item Suppose to define the problem as an SCSP:
	\begin{itemize}
		\item a partition in $Part(I)$ can be represented as an assignment of variables $I$
		\item the solution is the probabilty $sol(\Pi)$ associated to a partition $\Pi$
	\end{itemize}
\medskip
	\item The solution in the SCSP case, for any two networks, is 
	$val_{N_1}(\Pi_1) \otimes val_{N_2}(\Pi_2)$
	\begin{itemize}
		\item The solution considers only the probabilities caused by the same $\Pi$ on the
		two subnetworks
	\end{itemize}
	\item This is not compatible with the given definition of parallel composition
\end{itemize}
\end{frame}

\section{References}

\begin{frame}
\frametitle{References}
\begin{itemize}
	\item \emph{U. Montanari, M. Sammartino, A. Tcheukam} - Decomposition Structures for
	Soft Constraint Evaluation Problems: An Algebraic Approach
\medskip
	\item \emph{F. Bonchi, L. Bussi, F. Gadducci, F. Santini} - Polyadic Soft Constraints
\end{itemize}
\end{frame}

\section{Conclusions}

\begin{frame}
\frametitle{Conclusions}
\begin{itemize}
	\item We saw an algebraic framework for defining and solving SCEPs
\medskip
	\item We also saw as polyadic soft constraints can be seen as an s-algebra or
	a w-algebra
\medskip
	\item Using these frameworks, we can apply dynamic programming to SCSPs, thus improving
	their complexity
\end{itemize}
\end{frame}

\end{document}